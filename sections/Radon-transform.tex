\section{Radon transform analysis}
\label{sec:methods-II-Radon}

\subsection{Motivation}
\label{sec:motivation}

Many galaxies display kinematic features in their morphology such as warps, kinematically decoupled cores (KDCs), bars, oval distortions, tidal tails etc. The global kinematic $\Delta$PA${k}$ is single number value, representing a broad average for the the entire galaxy. Finer details of any radial variation across a velocity field can be obtained using the \textbf{Radon transform} method, details of which are presented below. In addition, PSBs have depleted their gas and star formation has shut down. The differential kinematic position angles $\Delta$PA${k}$ cannot be easily measured in gas-poor galaxies like our PSBs. However the Radon transform method can be applied to either or both the gas and stellar velocity fields. For our purposes we apply the Radon transform to the stellar velocity maps (only) of mainly gas-poor PSBs. 

\subsection{Description of the Radon transform}

The Radon transform (RT) is a mathematical transformation of an image devised to reveal internal properties of an object such as structure. The transform method was devised by Johann Radon in 1917 \citep{radon1917determination}. Since then the technique has been applied to many fields, including various branches of astronomy such as microwave, radio and x-ray applications. \citet{deans2007radon} describes many of these applications. Deans' book also includes an English translation of Radon's original German text. In his PhD thesis \cite{7910dc8d5b654c90ac4bc94c67d06f01} applied Radon transforms to the field of digital signal processing and presented algorithms for their implementation. For our present purposes \cite{2018MNRAS.480.2217S} describe the application of the Radon transform method to galaxy kinematic studies. Stark developed IDL\footnote{\href{https://www.harrisgeospatial.com/Software-Technology/IDL}{https://www.harrisgeospatial.com/Software-Technology/IDL}} (Interactive Data Language) code routines to apply the Radon transform to analyse stellar and gas velocity fields obtained from the MaNGA integral field survey. The objective is to quantify radial and other spatial variations in the kinematic position angles (PA$_{k}$).

\begin{figure}
    \centering
    \includegraphics[width=\columnwidth]{images/RadonPlots/Radon-transform-Stark.png}
    \caption[The Radon transform and its coordinate system]{The Radon transform: illustration of the Radon transform and its coordinate system taken from Figure 1 in \citet{2018MNRAS.480.2217S}. Line integrals across a 2-D image in x,y coordinate space are calculated along all possible lines, parameterised by the coordinates [$\theta$, $\rho$]. The left panel shows examples of two solid lines, L$_1$ and L$_2$, which are mapped through the Radon transform function to points $\theta_1$, $\rho_1$ and $\theta_2$, $\rho_2$ in [$\theta$, $\rho$] parameter space in the right-hand panel. Credit: Stark et al. 2018.}
    \label{fig:RadonTransform}
\end{figure}

Mathematically the Radon Transform, $R$, is defined as
\begin{equation}
    \label{eqn:radon}
    R(\rho,\theta)=\int_{L}{v(x,y)\, \diff l},
\end{equation}

where $v(x,y)$ is a 2D velocity field defined in Cartesian coordinates and $\int_{L}$ is the line integral at transformed sky-plane polar coordinates ($\rho,\theta$). This transform is illustrated graphically in Figure \ref{fig:RadonTransform} where lines through an image in 2D velocity space are mapped to a series of points in Radon transform [$\theta,\rho$] space by means of line integrals at various angles about the velocity space axes, and at various radial distances offset from the velocity space origin. 


\citet{2018MNRAS.480.2217S} apply a modification to the Radon transform, to obtain the \textit{Absolute} Radon Transform, as defined in Equation (\ref{eqn:radon}), by taking the integral of the absolute values of the velocity field difference of each point $v(x_i,y_i)$ and the mean of all values along the line segment.

\begin{equation}
    \label{eqn:radon_absolute}
    R_A=\int{| v(x,y) - \langle v(x,y) \rangle | \, \diff l}.
\end{equation}

There is another variant to the Absolute Radon Transform described by \citet{2018MNRAS.480.2217S} named the \textit{bounded}, or aperture restricted, Radon transform, R\textsubscript{AB}. Briefly, R\textsubscript{AB} involves placing integration limits $(\pm{r_{ap}})$, known as the Radon aperture, on the integral of Equation \ref{eqn:radon_absolute} in order to limit the number of spaxels across a velocity map that will be used in the integration. \citet{2018MNRAS.480.2217S} developed a suite of IDL algorithms to calculate the Radon transform on an input array representing a velocity field. The Radon transform IDL code is available on GitHub\footnote{\href{https://github.com/dvstark/radon-transform}{https://github.com/dvstark/radon-transform}}. 
The graphical output of the Radon transform code as applied to a synthetic velocity field is shown in Figure \ref{fig:Radon}.

\begin{figure}
    \centering
   	\includegraphics[width=\columnwidth]{images/RadonPlots/example.png}
    \caption[Model velocity field Radon transform plots]{Example of Radon transform code output applied to a model velocity field. The plots are obtained by running the Radon example code which is publicly available from David Stark's GitHub repository. The left panel shows a synthetic uniform velocity field model, the middle panel shows the the absolute Radon transform of the velocity field and the right panel shows the aperture-restricted absolute transform. We are concerned with the latter, the absolute aperture-restricted Radon transform of stellar and gas velocity fields in this work. Credit: David Stark.}
    \label{fig:Radon}
\end{figure}

Following \citet{2018MNRAS.480.2217S} we demonstrate the application of the Radon transform to stellar and gas velocity field data obtained from the MaNGA integral field survey for a selection of CPSB and RPSB galaxies. Figure \ref{fig:RT_8131-6101} shows the Radon transform output for the stellar (left) and gas (right) velocity fields of the central-type PSB 8131-6101. The transform of high velocity regions are displayed in green and low velocity areas in purple. The stellar velocity Radon transform plot (left) reveals a well defined velocity minimum track across the radial (vertical) coordinate, again shown in purple. Notably the right hand panel showing the gas velocity transform reveals few features, probably due to a sparsity of gas in CPSB 8131-6101.

\begin{figure}
    \centering
   	\includegraphics[width=\columnwidth]{images/RadonPlots/RT-snips/CPSB-8313-6101-RT-snip.png}
    \caption[Example of basic Radon transform plots for the gas and velocity fields of CPSB 8131-6101]{Bounded absolute Radon transform plots of the stellar velocity field (left panel) and gas velocity field (right) of the CPSB MaNGA PLATEIFU 8131-6101. The radial coordinate $\rho$ is on the vertical axis and the angular coordinate $\theta$ on the horizontal axis.}
    \label{fig:RT_8131-6101}
\end{figure}

% %% RT-plots.tex

\begin{figure}
    \centering
    \includegraphics[width=0.8\columnwidth]{images/RadonPlots/RT-snips/CPSB-8313-6101-RT-snip.png}
    \includegraphics[width=0.8\columnwidth]{images/RadonPlots/RT-snips/CPSB-9494-3701-RT-snip.png}
    \includegraphics[width=0.8\columnwidth]{images/RadonPlots/RT-snips/CPSB-8398-6102-RT-snip.png}
    \caption{CPSBs: Radon transforms of stellar velocity field (left) and gas velocity field (right) maps. From the top CPSB-8313-6101, CPSB-9404-3710 and CPSB-8398-6103.}
    \label{fig:CPSB-RTs}
\end{figure}

\begin{figure}
    \centering
    \includegraphics[width=0.8\columnwidth]{images/RadonPlots/RT-snips/RPSB-9872-3701-RT-snip.png}
    \includegraphics[width=0.8\columnwidth]{images/RadonPlots/RT-snips/RPSB-8932-12704-RT-snip.png}
    \includegraphics[width=0.8\columnwidth]{images/RadonPlots/RT-snips/RPSB-8554-3701-RT-snip.png}
    \includegraphics[width=0.8\columnwidth]{images/RadonPlots/RT-snips/RPSB-8323-6103-RT-snip.png}
    \caption{RPSBs: layout is as per Figure \ref{fig:CPSB-RTs}, showing the Radon transform maps from the top for for RPSB-9872-3701, RPSB-8932-12704, RPSB-8554-3701 and RPSB-8323-6103.}
    \label{fig:RPSB-RTs}
\end{figure}

\begin{figure}
    \centering
    \includegraphics[width=0.6\columnwidth]{images/RadonPlots/RT-snips/CPSB-8313-6101-SG-AP-00.png}
    \includegraphics[width=0.6\columnwidth]{images/RadonPlots/RT-snips/CPSB-8313-6101-SG-AP-04.png}
    \includegraphics[width=0.6\columnwidth]{images/RadonPlots/RT-snips/CPSB-8313-6101-SG-AP-08.png}
    \includegraphics[width=0.6\columnwidth]{images/RadonPlots/RT-snips/CPSB-8313-6101-SG-AP-12.png}
    \caption{CPSBs: Absolute Radon transforms of stellar velocity field (left) and gas velocity field (right) maps for CPSB-8313-6101, but with varying aperture settings. From the top $r_{ap}=0$, $r_{ap}=4$, $r_{ap}=8$ and bottom $r_{ap}=12$. Note the poor definition in the gas velocity field transform plot, indicating a lack of gas in the galaxy.}
    \label{fig:CPSB-RT-Apertures}
\end{figure}

   %% DO WE NEED THIS?

At this point in the project a significant upgrade to the Radon transform code was released on GitHub. The underlying Radon transform code that was run to produce the transformed stellar and gas velocity fields for the CPSB 8131-6101 in Figure \ref{fig:RT_8131-6101} remain unchanged, however the software has been enhanced to produce more sophisticated graphical output. An example of the upgraded graphical output for one of the control sample galaxies MaNGA plateifu 8442-3704 is shown in Figure \ref{fig:8442-3704-complete}.

\begin{figure*}
    \centering
    \includegraphics[width=0.9\textwidth]{images/RadonPlots/RT-SNIPS-NEW/8442-3704-complete.png}
    \caption[Radon transform code output graphics for stellar velocity field of the MaNGA plateifu 8442-3704]{The Absolute Radon transform code output graphics for the stellar velocity field of the galaxy MaNGA plateifu 8442-3704. The left side text block provides some relevant data about the galaxy extracted from the MaNGA \texttt{drpAll} file. The quadrature arrangement of panels to the right show: top left - the SDSS IFU gri image cutout; top right - the stellar velocity field with kinematic position angle PA$_{k}$ superimposed (represented by the magenta line). The orange coloured  bar in the upper left indicates the Radon aperture size in spaxels; bottom left - the absolute value of Radon transform re-scaled from 0 to 1 (as displayed in the colour bar). The locus of the minimum of the transform is plotted in magenta; and bottom right: the Radon trace plot.}
    \label{fig:8442-3704-complete}
\end{figure*}

The additional graphical features in the new output format are:
\begin{itemize}
\item SDSS 3-colour gri image cutout
\item Stellar or gas velocity field map (selectable)
\item Kinematic PA$_{k}$ superimposed on the velocity map
\item Absolute bounded Radon transform (normalised)
\item The Radon transform trace profile in [$\rho$,	$\theta$] space
\end{itemize}
In addition some useful data for the subject galaxy is output in a text box on the left-hand side.

This Radon transform graphical output for this particular galaxy, MaNGA ID 1-419195 plateifu 8442-3704, reveals some interesting and particularly significant stellar velocity kinematic features. We now turn attention to analysis of the Radon transform output to identify significant kinematic features. We do this by classifying the galaxy in terms of its Radon profile. Classification of Radon transform profiles is discussed in depth in Section \ref{sec:Radon-classification}. 

\subsection{Radon profile classification procedure}
\label{sec:Radon-classification}

\cite{2018MNRAS.480.2217S} identified 5 commonly recurring patterns in the stellar and gas velocity field Radon transform profiles of their MaNGA galaxy sample. These patterns were used to classify the computed Radon trace profiles. In this project work we adopt the same classification approach for the Radon profiles of our PSB galaxies and their control samples. Simplified models of 4 of the trace profile classes are shown in Figure \ref{fig:class-models}. In addition an asymmetric profile class was identified. Detailed examples of the trace profile class types are presented in Figure 7 of \cite{2018MNRAS.480.2217S}. The salient features of these 5 Radon profile classes are listed below:

\begin{itemize}
    \item Constant, \textbf{Type-C} : Radon profile with relatively constant trace minimum angle $\hat{\theta}$ at all radii $\rho$.
    \item Inner Bend, \textbf{Type-IB} : Galaxies whose Radon profiles exhibit symmetrical variations of $\hat{\theta}$ beginning at $|\rho|=0$, then transitioning to a constant value. 
    \item Outer Bend, \textbf{Type-OB} : Galaxies with constant Radon trace angle $\hat{\theta}$  at small $|\rho|$ which transition to a different value at a greater radius. 
    \item Inner Bend + Outer Bend, \textbf{Type-IB+OB} : Galaxies with Radon profiles showing a combination of the features of Type-IB and Type OB profiles.
    \item Asymmetric, \textbf{Type-A} : The value of the $\hat{\theta}$ varies significantly with $\rho$ across opposite sides of the transform R\textsubscript{AB}. 
 \end{itemize}

\begin{figure}
    \centering
    \includegraphics[width=0.8\columnwidth]{images/RadonPlots/Radon-class-models.png}
    \caption[Radon profile class feature identification: toy models]{Toy models of the Radon profile (trace angle minimum $\hat\theta$ versus radius $\rho$). The sub-plots display examples of 4 of the Radon profile classes used for classification of the RT trace plots. Upper left: Constant, Type-C; upper right: Inner Bend, Type-IB; lower left: Outer Bend, Type-OB; and lower right: Inner Bend + Outer Bend, Type-IB+OB. Source: Stark et al. (2018) figure 8.}
    \label{fig:class-models}
\end{figure}

Mathematical functions describing these Radon profile classifications have been identified by \citet[][section 3.6]{2018MNRAS.480.2217S}. This has enabled code routines to be developed which can provide automatic classification of the Radon trace profiles. Results of the automatic classification routines were made available quite late on during the progress of the project work, and therefore were not used extensively in this analysis. Instead, we have adopted a simple visual classification method to categorise each of our sample galaxy into one of the 5 Radon profile trace types listed above. Visual classifications were determined by following a 3-step process:

\begin{enumerate}
    \item Firstly we obtain the MAPS data cube FITS files for the selected PSB galaxies listed in Tables \ref{tab:my-CPSBs} and \ref{tab:my-RPSBs}, and a similar number of 'normal' galaxies drawn from their respective control samples, as described in Section \ref{sec:controls}, downloaded from the MaNGA website.
    \item Next, we process each of data cubes through the Radon transform wrapper code to obtain graphical output files showing the galaxy SDSS $gri$ image cutout, the MaNGA stellar velocity map, the absolute bounded Radon transform R\textsubscript{AB} plot and the Radon profile plot of $\hat{\theta}$ versus $\rho$. An example of this output for the CPSB galaxy 8979-1902 is shown in Figure \ref{fig:CPSB-8979-1902-SNIP}. 
    \item  We then examine the output plot for each galaxy to  visually assess the relative qualitative strength of each of the 5 classification features by assigning a numeric weighting as given in Table \ref{tab:features}. This method adds a semi-quantitative approach to the visual assessment process.
\end{enumerate}

\begin{figure*}
    \centering
    \includegraphics[width=0.8\textwidth]{images/RadonPlots/RT-SNIPS-NEW/CPSB-8979-1902-SNIP.png}
    \caption[Radon transform code output graphics for the CPSB 8979-1902]{Radon transform code output graphics for the CPSB 8979-1902. Top left: SDSS gri image cutout, Top right: stellar velocity map with kinematic position angle (PA$_{k}$) shown as the magenta line, lower left: Radon transform (RT) of the stellar velocity field with the transform minimum angle $\hat\theta$ plotted across radius $\rho$ in magenta, lower right: the Radon profile trace of the RT minimum.}
    \label{fig:CPSB-8979-1902-SNIP}
\end{figure*}

\begin{table}
    \caption[Relative weighting of Radon profile feature strengths used in visual classification]{Relative weighting of Radon profile feature strengths used for the visual classification of Radon profile feature types: Constant, Inner Bend, Outer Bend, Inner Bend + Outer Bend or Asymmetric. The weighting is assigned to help quantify the visual appearance of the trace profile plots.}
    \label{tab:features}
    \centering
    \begin{tabular}{cl}
    \hline
    Value & Visual appearance \\
    \hline
    2 & The feature is visually predominant \\
    1 & Some evidence of the feature is apparent \\
    0 & The feature is absent \\
    \hline
    \end{tabular}
\end{table}

The Radon output graphic plots for 127 galaxies (PSBs and some of control sets) were visually assessed in alphanumerical order of their PLATEIFU file name tag, which effectively creates a random sequence across the groups. Classifying the galaxies in each group sequentially may have led to preferential identification of similar features in that group, thereby introducing a classification bias.

The Radon transform output plots for each galaxy were inspected to make an assessment of the visual strength of each of the Radon profile type features evident. A numerical value 0, 1 or 2, representing the visual strength of apparent features from Table \ref{tab:features}, was assigned to each of the 5 predefined Type classes for the galaxy. Based on the relative strength values allocated, a predominant feature Type (C, IB, OB, IB+OB or A) was assigned to that particular galaxy. To determine the relative predominance of Type-IB+OB features we simply summed the strength values assigned to Types IB and OB together. 
In many cases there was ambiguity in the absolute Type assessment, where we had difficulty to select only one of the 5 predefined classes. In these cases a secondary assessment was made, generally this was the most prevalent Type plus a sub-dominant feature. The secondary assessment was intended to be used later to refine the analysis process. 

As a demonstration of the Radon profile visual classification method we select the example of the spiral galaxy 8979-12701. The Radon transform and Radon profile trace for this galaxy are shown in Figure \ref{fig:OB+IB}. Comparing the Radon trace profile with the model traces in Figure \ref{fig:class-models} and the examples given in  Figure 7 of \cite{2018MNRAS.480.2217S}, the visual classification process identified the strength of the features as: C = 1, IB = 2, OB = 2, IB+OB = 2+2 = 4, and A=0. This trace profile is comparable to the Type-OB+IB model and consequently the Radon profile of this galaxy is categorised as Type-OB+IB.

The classification process outlined above was carried out independently by 2 persons in an endeavour to provide some means of eliminating personal subjectivity. The intent is similar  to that adopted by the Galaxy Zoo project which used large-scale public collaboration to classify galaxy morphology \citep{2017MNRAS.464.4176W}. The galaxy-by-galaxy Radon profile Type classifications determined by the two assessors are listed in Tables \ref{tab:full-visual-classification} and \ref{tab:visual-classification-B} in Appendix \ref{sec:visual-classification-tables}.

\begin{figure}
    \centering
    \includegraphics[width=\columnwidth]{images/RadonPlots/RT-SNIPS-NEW/8979-12701-VA-OB+IB.png}
    \caption[Example of the Radon profile visual classification of galaxy 8979-12701]{Example of the Radon profile visual classification method for galaxy 8979-12701. The Radon transform (RT) plot is shown in the left panel and the Radon profile trace on the right. The RT plot minimum (magenta line) shows an indication of a wide outer bend (OB) feature. The trace plot also shows a narrower inner bend (IB) feature centred at radius $\rho=0$. We therefore classify the Radon profile of this galaxy as Type-OB+IB.}
    \label{fig:OB+IB}
\end{figure}

During the classification assignment exercise  some difficulties were encountered mainly with Radon trace profiles that did not fit easily into on of the 5 classification Types. An example of this is seen in the case of 8555-3701 where a clearly defined Inner Bend appears superimposed on an asymmetric trace as shown in Figure \ref{fig:8555-3701-A+IB}. The form of this trace does not fall readily into either of the Type-A or Type-IB categories, however faced with a choice of Types, and a requirement to select only one of the 5 categories, the natural choice was to opt for the predominant feature, in this case Type-A, asymmetric. The reader may disagree and opt for Type-IB, or even Type-OB+IB. This demonstrates the challenges encountered in the visual classification of Radon profiles.

\begin{figure}
    \centering
    \includegraphics[width=\columnwidth]{images/RadonPlots/RT-SNIPS-NEW/8555-3701-A+IB.png}
    \caption[Radon transform and profile trace plots for the galaxy 8555-3701]{Radon transform and profile trace plots for the galaxy 8555-3701. An inner bend appears superimposed on a generally asymmetric trace.}
    \label{fig:8555-3701-A+IB}
\end{figure}

In many other cases bends, or detectable velocity field disturbances, are evident as notches at well off-centre radii on otherwise constant or largely asymmetric traces. To obtain a comprehensive census at this level of detail these sub-dominant and off-centre features should be taken into account in a secondary analysis as mentioned above. Other than presenting a listing of the mixed secondary classifications obtained by classifier A in Table \ref{tab:full-visual-classification}, we did not pursue a more detailed secondary analysis in this present work.









