\subsection{Future work}
\label{sec:future-work}
As with all studies, our results and conclusions would benefit from an increased sample size. We focused on a total of 68 PSBs from MaNGA MPL-6 data. MPL-8 should encompass a larger sample of PSBs. A similar analysis to that performed in this study using a larger sample of PSBs would improve the statistical significance of the results.

While carrying out visual classification of Radon trace plots we noted that many edge-on galaxies have poor quality stellar velocity maps. These low resolution maps generated noisy Radon traces which were particularly difficult to classify by eye. We recommend to take a cut of PSBs and control galaxies that present a fuller face-on aspect thereby excluding edge-on dusty galaxies. We would expect that the Radon profile features which could indicate merger signatures to be detected more readily from the face-on sample. A larger number of PSBs from MPL-8 would present the opportunity to make this selection cut based on inclination.

We recommend to investigate the disparity in Radon profile type classification further. Firstly by obtaining a complete set of automated classifications for our full sample of PSBs and controls, and repeat the statistical analyses that we carried out on the visual classifications. We should also extend the visual classification process by engaging additional classifiers. We can then investigate why there are inconsistencies in Radon profile type classifications by either visual or automated methods. 

In another area for future study we should investigate any correlation between Galaxy Zoo morphology classifications and the Radon profile trace types as discussed this present study. It is envisaged that the catalogue of \citet{2018MNRAS.479..415A} will be useful here as there study was specifically targeted at mergers. If a correlation is apparent it would enable us to refine the guidelines for our visual classification procedure and deliver more consistent results from diverse classifiers.


