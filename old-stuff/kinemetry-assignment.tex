
\subsection{kinemetry}
Kinemetry analysis employs the \texttt{kinemetry} software package developed by \citet{2006MNRAS.366..787K} to distinguish disc dominated systems from those exhibiting major mergers. The \texttt{Kinemetry} method involves mapping the gas velocity field and the gas velocity dispersion. Classification of a galaxy as a disc system or merger depends on the relationship between the gas velocity $v$ and the gas velocity dispersion $\sigma$.

\citet{2016A&A...591A..85B} examine the gas kinematics of nearby (ultra)luminous infrared galaxies ((U)LIRGs) at $z<0.1$. The objective is to analyse the kinematic properties of local (U)LIRGs to characterise their structures and thereby classify those (U)LIRGs as having disc structures (disc class), or displaying evidence of major merger activity (merger class). Their method employs optical integral field spectroscopy (IFS) data obtained at the VLT. H$\alpha$ emission is used as a gas velocity tracer. \citet{2016A&A...591A..85B} conclude that their results confirm that well-defined discs can be effectively distinguished from well-defined mergers but there is intermediate, indeterminate class. They note that the \texttt{kinemetry} method is sensitive to angular resolution of the integral field unit (IFU). \citet{2008ApJ...682..231S} had earlier performed an  analysis of warm gas kinematics as traced by H$\alpha$ emission, but concentrated on sample at $z\sim2$ using the NIR IFS instrument SIMFONI on the VLT. 

\subsection{Tilted ring fitting}
\label{sec:tilted-ring-fitting}

[TODO: write up some details of this method.]

The tilted ring fitting method enables kinematic features in disc galaxies, such as oval distortions and warped discs, to be quantified  \citep[see e.g.][]{1978PhDT.......195B,1981AJ.....86.1825B,2007A&A...468..731J}
