\subsection{Mergers and morphology}
\label{sec:mergers}

The term morphology refers to the structural features of galaxies. Commonly observed morphological features include discs, bulges, bars and spiral arms. These features form the basis of the Hubble classification system. A more complete classification system could also include other morphological features such as rings, warped discs, halos, tidal tails, arms and bridges. Study of  morphological structure provide insights into galaxy evolution. For a review of morphological types and their link to galaxy evolution see \cite{2011arXiv1102.0550B}.

Here we present a summary review of the literature on the detection of mergers and post-coalescence systems, or post-mergers, using observations of morphology. Mergers play a key role in galaxy evolution. In order to better understand the consequences of the merger process on evolution we need to employ refined methods to detect the morphological signatures of mergers. \cite{2016MNRAS.456.3032P} describe the morphological indicator designated 'shape asymmetry' for automated identification of galaxies exhibiting faint asymmetric tidal features indicative of ongoing or past mergers in order to determine whether PSBs play a transitory role in the buildup of the red sequence. \cite{2011arXiv1102.0550B} have developed numerical simulations to explore the sensitivity of galaxy mass ratio in the detection of major mergers in starburst galaxies. Adopting a similar but more extensive approach, \cite{2019ApJ...872...76N} developed a merger classification scheme that can be applied directly to SDSS images. Their method is based on hydrodynamical and N-body models of mergers, which were then trained on model SDSS images. They find their method is sensitive to mass ratio, and major mergers are also sensitive to asymmetry. In a progression of that earlier work  \cite{2019DDA....5020304N} have extended the SDSS image classification method to incorporate kinematic predictors derived from MaNGA stellar velocity maps. This is the first merger classification scheme that utilises both imaging and kinematics. The authors will apply the technique to explore how star formation rates change with different stages and types of merger. 

In this study, however, we are interested in the identification of past merger events, or post-mergers. Such major may have been the cause of star-formation quenching resulting in our transitional PSBs. The automated merger classification method of  \cite{2019DDA....5020304N} is expected to be of significant value in the context of identifying post-merger morphological features in the study PSB galaxies and their role in galaxy evolution. 


